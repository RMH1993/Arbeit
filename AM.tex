\section{Anforderungsmanagement}

Das Anforderungsmanagement, ist eine Management Disziplin die das Software Engineering unterstützt. Es setzt sich damit auseinander eine optimale Lösung durch strukturierte und organisierte Planung und Definition der Anforderung zu erreichen.

Durch das Anforderungsmanagement soll vermieden werden, dass durch schlechte, unzureichende oder missverständliche Definitionen, eine Entwicklung am Ende nicht dem Entspricht, was der Auftraggeber haben wollte.

Das Anforderungsmanagement lässt sich in unterschiedliche Arbeitspakete unterteilen, hierzu lassen sich in der Literatur die unterschiedliche Ausprägungen und Anzahlen finden. Bei Marcus Grande\autocite[10]{100minAM} wird in vier Haupttätigkeiten unterteilt. Für die Diplomarbeit habe ich mir beispielhaft die Punkte ausgesucht, wie sie im Buch "Anforderungsmanagement in sieben Tagen"\autocite[30-31]{AMin.sieben.T} zu finden sind. Es gibt 


\subsection{Arbeitspakete im Anforderungsmanagement}
\begin{itemize}
\item Anforderungsmanagement planen \\
		Das Anforderungsmanagement dient der Planung und so ist es auch nicht weiter
		verwunderlich, dass auch in der Vorbereitung eine Planung für das Anforderungsmanagement
		sinnvoll ist. Es stehen eine Vielzahl von Tools, welche das Anforderungsmanagement
		unterstützen, die Art und weise der Dokumentation sollte festgelegt werden und die
		Anforderungsattribute sollten festgelegt werden. Bei den Attributen muss auf die fünf
		Pflichtattribute für Anforderungen geachtet werden.\autocite[40]{100minAM}

\item Ausgangspunkt finden \\
		\textit{Um das Ziel zu finden, sollte man wissen wo man los geht} \\
		Bevor man mit der Zielfindung beginnen kann, sollte man wissen, wie der aktuelle Stand 
		ist. Man kann das mit einem Orientierungsmarsch bei der Bundeswehr vergleichen, bei dem 
		Soldaten mit einem Kompass und einer Geländekarte ein Ziel erreichen müssen, wenn sie 
		nicht wissen wo sie los gehen, ist es unwahrscheinlich, dass sie das ausgegebene Ziel 
		erreichen. Wie schaut der zu bearbeitende Prozess aus und wer ist darin Akteur und/oder 
		Betroffener. Durch das ermitteln des IST-Zustand kann man seine Stakeholder ermitteln, 
		lernt den Prozess im aktuellen Zustand, seine Grenzen und die von außen einwirkenden 
		Elemente kennen. 

\item Anforderungen erheben\\
		Für die Erhebung von Anforderungen existieren verschiedene Methoden und Quellen. 
		
		\textbf{Quellen} für Anforderungen sind 
		\begin{itemize}
		\item Stakeholder, die Aufgrund ihrer Mitarbeit am Prozess oder durch Auswirkungen und 
		Ergebnissen dieses, betroffen sind.
		\item Prozessmodelle, in denen der Ablauf des Prozesses dokumentiert ist
		\item Beobachtungen des Prozessalltags
		\item Prozessbeschreibungen oder Problemberichte
		\end{itemize}		
		\textbf{Methoden} zur Erhebung von Anforderungen sind
		\begin{itemize}
		\item Interviews mit Stakeholdern oder Verantwortlichen
		\item Workshops mit Stakeholdern 
		\item Analysen von Berichten und Beschreibungen
		\end{itemize}
		
		Je nach zur Verfügung stehender Quelle ist eine dazu passende Methode zur Erhebung 
		dieser anzuwenden.
		
\item Anforderungen dokumentieren\\
		Mit dem Erheben von Anforderungen wird auch das dokumentieren dieser zu einem Thema. 
		Wenn die Disziplin des Anforderungsmanagement in einem größeren Unternehmen schon weit
		fortgeschritten ist wird es für die Dokumentation evtl. schon geeignete Tools im 
		Unternehmen geben. Bei kleinen Firmen wird dies in der Regel wohl nicht der Fall sein.
		Für die Anforderungsdokumentation stehen verschiedene Methoden, Darstellungsformen und 
		Tools zur Verfügung. Diese dienen zum einen der besseren Visualisierung der 
		Anforderungen, evtl. der Gestaltung beispielhafter Oberflächen als Entwurfsmuster 
		können aber im Falle einer rein Verbalen Funktionsbeschreibung auch in Form einer 
		Tabelle sein. Wichtig bei der Dokumentation ist immer, dass man einen Mix aus den zur 
		Verfügung stehenden Methoden wählt um alle Aspekte, die durch die Anforderungen erhoben 
		werden auch passend und widerspruchsfrei abbilden zu können
		\autocite[93-137]{AMin.sieben.T}. Es ist darauf zu achten, dass man sich mit dem gewählten 
		gut auskennt um das gesamte Potenzial ausschöpfen zu können und bei der Gesaltung darauf 
		achten das auch die beteiligten Stakeholder das Ergebnis verstehen.
		
		
\item Anforderungen qualitätssichern
		Nachdem alle Anforderungen zusammengetragen wurden, müssen diese vom Anforderungsmanager 
		auf Integrität, Vollständigkeit und Widersprüchlichkeit überprüft werden. In der Regel ist 
		eine Iteration der vorangegangen Schritte notwendig um Lücken zu schließen und 
		Widersprüchlichkeiten auszumerzen. Hierzu sind Interviews und erneute Workshops ein 		
		probates mittel. Am Ende muss eine Lücken freie Prozessbeschreibung durch die Anforderungen vorliegen
		und alle Stakeholder diese Anforderungen unterschreiben.
	
\item Anforderungen verwalten
		Im Buch 100 Minuten für Anforderungsmanagement\autocite[103]{100minAM} wird dieser Punkt als
		Pflege und Verwaltung von Anforderungen bezeichnet, was auch zutreffender ist. Denn es
		handelt sich hierbei nicht nur um ein reines Verwalten der Anforderungen. 
			
		Anforderungen können sich innerhalb eines Projektverlaufs ändern, es können Anforderungen hinzu
		kommen oder auch weg fallen. Hierbei ist es immer notwendig, dass jede Veränderung dokumentiert 
		wird und die Anforderungen Versioniert sind, dadurch hat man bei Terminverschiebungen etwas in 
		der Hand, da durch verändern oder erweitern der Anforderungen auch immer eine Verschiebung der 
		Termine möglich sein kann. Neben dem Namen des ändernden, einem Datum und einer Versionsnummer 
		sollte auch immer eine Erklärung, was geändert wurde, mit erfasst werden. Stakeholder sollten 
		Zugriff auf die verwalteten Dokumente haben um jederzeit noch einmal nachlesen zu können, wie 
		die	Anforderungen getroffen wurde. Wenn es in der Entwicklungsphase zu viele Änderungen in den
		Anforderungen gibt, läuft man Gefahr, dass das Projekt zu sehr in die Länge gezogen wird und im
		Extremfall zu scheitern droht. Es ist darauf zu achten, dass die Änderungen mit der Zeit abnehmen
		und ab einem gewissen Zeitpunkt stabil sind und keine zusätzlichen Forderungen oder Änderungen mehr 
		hinzu kommen.
		
\end{itemize}

\section{Anforderungen}

Anforderungen unterliegen verschiedenen Attribute, Arten und bedürfen bestimmter Formulierungen, damit sowohl Anwender als auch Entwickler unmissverständlich das gleiche aus ihr ableiten können.

\subsection{Definition der Anforderung}

In der Literatur finden sich die unterschiedlichsten Definitionen für Anforderungen und so kommt auch eine Analyse des KIT zur Schlussfolgerung, dass keine von ihnen als allgemein gültig betrachtet werden kann\autocite[9][2.3.1]{KPAvMdAM}. Es lässt sich dabei festhalten, dass eine Anforderung immer etwas definiert, was man von einem Produkt erwartet und diese vorzugsweise in irgendeiner schriftlichen Beschreibung festgehalten und dokumentiert werden sollte.

\subsection{Arten der Anforderungen}

Anforderungen lassen sich in viele unterschiedliche Arten unterteilen. Das hängt auch mit dem jeweiligen Projekt zusammen. So gibt es Arten von Anforderungen die immer auftreten und andere, die vorkommen können oder auch nicht.  
Bei der Unterteilung der Anforderungen ist man sich in der Literatur auch nicht immer einig. Während die einen als grobe Trennung in Anwenderanforderungen und Systemanforderungen unterteilen\autocite{AM} wählen die anderen Funktionale-Anforderungen und Nicht-funktionale Anforderungen\autocite{100minAM}\autocite{AMin.sieben.T}. es werden die Anforderungen bei beiden Varianten noch in weitere Unterkategorien unterteilt.

\begin{itemize}

\item Funktionale- \& Nicht-Funktionale Anforderungen\\
Bei der Unterteilung in Funktionale und Nicht-funktionale Anforderungen werden alle Anforderungen, die direkt die Funktion und ihre Auswirkung beschreiben in Gruppe der Funktionalen Anforderungen einsortiert und alle anderen Anforderungen in die Kategorie der Nicht-funktionalen Anforderungen. Diese werden dann noch in weitere Kategorien unterteilt. 

\item Anwender- \& Systemanforderungen\\
Bei der Gliederung in Anwenderanforderungen und Systemanforderungen, werden in die Kategorie der Anwenderanforderung alle einsortiert die beschreiben wie eine Software für den jeweiligen Anwender bedienbar sein sollte, sie beschreiben ein konkretes Problem\autocite{AM}, Systemanforderungen beschreiben wie die Lösung eines Problems umgesetzt werden soll, was es an Schnittstellen zu anderen Prozessen gibt, wie die Struktur von Daten sein sollen.
Anwenderanforderungen werde in der Regel durch die späteren Nutzer erhoben. Unterschiedliche Nutzergruppen, werden auch unterschiedliche Anforderungen hervor bringen, da sie den Prozess aus ihrer eigenen für sie erforderlichen Sicht betrachten und entsprechend die Anforderungen erheben. Systemanforderungen werden dagegen eher von den Entwicklern erhoben, die das System als ganzen betrachten müssen, was sind an Schnittstellen zu anderen Prozessen gegeben, wie können bestimmte Anforderungen umgesetzt werden, dass vom gewünschten Input auch der vom Anwender geforderte Output erzeugt wird. 


\end{itemize}

Es ist für den Erfolg eines Projektes nicht ausschlaggebend, an welche Art der Unterteilung man sich hält.
Wichtig dabei ist, dass man seine Anforderungen in irgendeiner Art und weise unterteilt und so die in großer Stückzahl auftretenden Anforderungen strukturiert. 

\subsection{Attribute von Anforderungen}

Jede Anforderung muss bestimmte Attribute besitzen. Sie dienen zum einen der eindeutigen Unterscheidung, Beschreibung und der Rückführbarkeit. Dabei sollte beachtet werden, dass so viel Attribute wie erforderlich gewählt werden aber eben auch so wenig wie möglich. Denn Attribute müssen auch gepflegt werden und der Aufwand an Pflege und die Fehlermöglichkeit nehmen mit steigender Anzahl an Attributen zu.

\subsection{Methoden zum Erheben von Anforderungen}

\subsection{Dokumentieren von Anforderungen}


% Das Anforderungsmanagement, dient dazu in einem Projekt das gewünschte Ergebnis schon zu beginn  durch eine

