\section{Anforderungsmanagement}

Das Anforderungsmanagement, ist eine Management Disziplin, welche sich damit auseinandersetzt eine optimale Lösung durch strukturierte und organisierte Planung und Definition der Anforderung zu erreichen.

Durch das Anforderungsmanagement soll vermieden werden, dass durch schlechte, unzureichende oder missverständliche Definitionen, eine Entwicklung am Ende nicht dem Entspricht, was der Auftraggeber haben wollte.

Das Anforderungsmanagement lässt sich in unterschiedliche Arbeitspakete unterteilen, hierzu lassen sich in der Literatur die unterschiedlichsten Ausprägungen und Anzahl finden. Für die Diplomarbeit habe ich mir beispielhaft die Punkte ausgesucht, wie sie im Buch "Anforderungsmanagement in sieben Tagen"\autocite[30-31]{AMin.sieben.T} zu finden sind.

\begin{itemize}
\item Anforderungsmanagement planen \\
		Das Anforderungsmanagement dient der Planung und so ist es auch nicht weiter
		verwunderlich, dass auch in der Vorbereitung eine Planung für das Anforderungsmanagement
		sinnvoll ist.

\item Ausgangspunkt finden
\item Anforderungen erheben
\item Anforderungen dokumentieren
\item Anforderungen qualitätssichern
\item Anforderungen verwalten
\end{itemize}

% Das Anforderungsmanagement, dient dazu in einem Projekt das gewünschte Ergebnis schon zu beginn  durch eine

