\section{Besonderheiten der Industriellen Fertigung}
Durch den Einsatz der Messmaschinen in der industriellen Produktion wird an diese und an die Software, welche diese betreibt einiges an Anforderungen gestellt, welcher Software, die im Bereich einer Verwaltung oder Prozessteuerung eingesetzt wird nicht unbedingt unterliegt.

\subsection{Rückführbarkeit}

Messgeräte haben die Vorgabe, dass die mit Ihr ermittelten Messergebnisse Rückführbar sind. Das heißt, die Anbieter von Messgeräte sind dazu verpflichtet, die Messgeräte nach bestimmten Normen zu prüfen und müssen dafür speziell kalibrierte Prüfmittel verwenden und die in Software vorkommenden Berechnungs- und Filtermethoden müssen entsprechend auch zertifiziert sein. In der Software des Controller, sind ebenfalls Berechnungsmethoden z.B. nach Gauss und Filter, für die Filterung von Messpunkten enthalten. Zusätzlich müssen für die Fehlerkorrektur der Maschine (CAA), entsprechende Modelle berechnet werden und diese über die Abnahmen der Messgeräte verifiziert werden.

\subsection{Anforderung an die Wartbarkeit der Systeme}

Bei Messgeräten handelt es sich nicht um kleine Investitionen für ein Unternehmen. Entsprechend sind diese auch immer über recht lange Zeiträume im Einsatz. Dabei sind 20 und mehr Jahre keine Seltenheit. Nach Investitionen für bestehende Maschinen sind immer schwierig, daher muss es mit den Systemen möglich sein, auch Updates und Erweiterungen integrieren zu können ohne das weitere Investitionen getätigt werden müssen. Dies birgt im Umkehrschluss jedoch auch das Risiko, dass man über einen sehr langen Zeitraum die Systeme abwärtskompatibel gestalten muss. So ist es keine Seltenheit, das man bei Kunden noch PC-Systeme mit Windows 98, 2000 oder NT antrifft. Diese Systeme müssen jedoch auch weiterhin gepflegt und supportet werden, soweit dies möglich ist. Sollten Hardwarekomponenten des Rechners getauscht werden müssen, ist ein Upgrade auf ein aktuelles System unumgänglich. In diesem Fall ist mit aktuell jedoch nicht immer das neuste zur Verfügung stehende Betriebssystem gemeint. Als Lieferant von Maschinen, muss man sich immer auch an sehr viele Vorgaben seitens der Kunden halten. So hat zum Beispiel das derzeit aktuelle Betriebssystem Windows 10 von Microsoft in der Industrie noch nicht wirklich Einzug erhalten. Dies ist keine Seltenheit, einige Betriebssysteme haben es in ihrem kompletten Produktlebenszyklus nie wirklich oder nur sehr vereinzelt in Industriebetriebe geschafft.

\subsection{Reaktionszeiten im Fehlerfall}

Standzeiten von Maschinen kosten die betreibenden Unternehmen viel Geld. Daher gibt es an Lieferanten der fertigenden Industrie, sei es als Zulieferer oder als Maschinenlieferant immer die Anforderung, dass Stillstandszeiten so gering wie irgend möglich gehalten werden. So sind Verträge, in denen Maschinenlieferanten Forderungen von Reaktionszeiten innerhalb 12h und Reparaturzeiten von maximal 24h keine Seltenheit. Größere Betreiber legen sich auch gerne verschiedene Komponenten auf Lager um im Bedarfsfall schneller reagieren zu können. Gleichzeitig ist der Lieferant gefordert, Servicepersonal bereitstellen zu können, die entsprechend schnell agieren können. Datensicherungen der beim Kunden installierten Controller Versionen mit allen Konfigurations- und Korrekturdaten sind daher auch beim Lieferant gespeichert, um evtl. neu benötigte Rechnersysteme schneller aufgesetzt und Einsatzbereit zu bekommen.
