\chapter{theoretischer Teil}
\section{Versionsverwaltung}
\subsection{Anfänge der Versionsverwaltung}
Die Anfänge der Versionsverwaltung reichen bis in die 70er Jahre des vergangenen Jahrhunderts zurück. Anfang der 70er wurde bei dem amerikanischen Telekommunikationsunternehmen AT \& T eines der ersten Versionsmanagement System entwickelt, Source Code Control System. Es war für die Verwaltung einzelner Dateien ausgelegt, die nur von einem Entwickler bearbeitet wurden. Ganze Projekte und Entwicklung im Team konnte mit diesem System nicht verwaltet werden. 
Etwa 10 Jahre später wurde an der Purdue University durch Walter F. Tichy das Versionierungstool RCS entwickelt. Dabei handelte es sich wieder um ein Dateien verwaltendes Versionierungsprogramm. Dies war bereits für die Entwicklung im Team ausgelegt, vom Aufbau her jedoch so, dass wenn ein Entwickler an einer Datei gearbeitet hat diese für den Schreibzugriff durch andere Entwickler blockiert war. Beide Systeme hatten gemein, dass sie für die Versionsverwaltung lokaler Dateien waren.
Ein Jahr nach der Entwicklung von RCS wurde mit der Entwicklung von CVS (1986) begonnen. Ursprünglich als Servererweiterung für RCS gedacht, wurde das zuerst auf UNIX Shell-Skripten basierende System im Jahre 1989 in C neu programmiert. CVS verfolgte einen anderen Ansatz als seine Vorgänger, die Dateien und Projekte waren zentral auf einem Server abgelegt und die einzelnen Entwickler konnten sich ihre Arbeitskopien von der herunter laden. So konnten auch mehrere Entwickler an der gleichen Datei arbeiten. Auch war CVS in der Lage Verzeichnisstrukturen, wenn auch noch sehr eingeschränkt, zu verwalten. 
Viele der später entwickelten System bauen auf diesen Systemen auf.





\subsection{Grundlagen der Versionsverwaltung}

Der Ursprüngliche Gedanke der Versionsverwaltung war, dass man damit erfolgte Änderungen in unterschiedlichen Versionen sichtbar und differenzierbar machen konnte. Es sollte möglich sein eine Historie der erfolgten Änderungen abzubilden und für den/die Entwickler sichtbar zu machen. Noch größere Bedeutung bekam das ganze als nicht mehr nur ein Entwickler ein Thema bearbeitete sondern mehrere Entwickler im Team arbeiteten. Die Entstehung von Open-Source-Projekten mit weltweit verteilten Entwicklern hat die Entwicklung entsprechender Versionierungstools beschleunigt und deren Funktionsumfang und Herangehensweise maßgeblich beeinflusst.
Die Funktionsweise der Systeme lässt sich in Lokale, Zentrale und Verteilte Versionsverwaltung unterteilen. 
\begin{itemize}
\item{Lokale Versionsverwaltung}\\
Bei der Lokalen Versionsverwaltung waren die zu versionierenden Dateien lokal auf dem Rechner des Entwickler abgelegt. In diesen Bereich fallen auch nur die beiden zuerst entwickelten Systeme SCCS und RCS. 


\item{Zentrale Versionsverwaltung}\\
Bei der Zentralen Versionsverwaltung werden die Daten bzw. Repositorien auf einem zentralen Server abgelegt und die Entwickler können sich von diesem Server eine lokale Arbeitskopie herunter laden und nach dem Bearbeiten wieder zurück spielen. Die Versionshistorie ist nur auf dem Server ersichtlich und wenn man eine Information aus einer früheren Version benötigt ist zwingend eine Verbindung zum Server notwendig. In diesen Bereich fallen zwei der am bekanntesten Open-Source-Systeme CVS und Subversion. Aber auch viele der kommerziellen Lösungen, die auf dem Markt sind, beruhen auf diesem System.



\item{Verteilte Versionsverwaltung}\\
Bei der Verteilten Versionsverwaltung besitzt jeder Entwickler lokal ein eigenes Repository des Projekts an dem er arbeitet. Die einzelnen Repository werden zwischen den unterschiedlichen Entwicklern immer wieder synchronisiert und somit alle Änderungen verteilt bzw.\@	 zusammengeführt. Ein zentraler Server ist hier nicht notwendig. Sehr häufig wird jedoch ein zentrales Repository eingesetzt um von dort aus das Produktivsystem zu bilden oder neuen Entwicklern einen Zentralen Punkt zu schaffen an dem sie sich ihre erste Arbeitskopie herunterladen können. Bei diesem System hat jeder Entwickler immer die komplette Versionshistorie mit in seiner Kopie des Repository und kann evtl. später entdeckte Bugs lokal in den früheren Versionen suchen und beheben.
\end{itemize}

Bei den unterschiedlichen Versionierungssystemen, gibt es auch unterschiedliche Konzepte, was die Arbeitsweise angeht. Man unterscheidet hier zwischen:
\begin{itemize}
\item Lock Modify Write\\
Bei dieser Arbeitsweise, erstellt sich der Entwickler eine Arbeitskopie der Datei, welche er ändern möchte. Das System sperrt in dieser Zeit die Datei für Schreibzugriffe anderer Entwickler. So können von einer Datei nie zwei unterschiedliche Versionen entstehen und auch keine Konflikte beim Zusammenführen der Dateien. Der Nachteil daran ist, dass nur ein Entwickler an der Datei arbeiten kann und ein anderer, der evtl. dringende Änderungen einfügen muss, dies erst tun kann, wenn der erste Entwickler die Datei wieder freigegeben hat. Ein weiterer Nachteil, wenn eine Datei gesperrt war und es bei dem Entwickler der die Datei gesperrt hatte, einen Systemverlust gab, war die Sperre noch immer vorhanden und musste umständlich entfernt werden. Bekannteste Vertreter für diese Arbeitsweise sind RCS und VSS von Microsoft. Bei den Verteilten Versionsverwaltung ist diese Arbeitsweise ausgeschlossen.

\item Copy Modify Merge\\
Bei dieser Arbeitsweise, erstellt sich jeder Entwickler seine lokale Arbeitskopie und entwickelt in dieser, hat er einen Teil fertig gestellt und will diesen den anderen zur Verfügung stellen oder ihn in das Haupt Repository übertragen, führt er einen Merge aus. Das heißt, die von ihm geänderte Datei wird auf den Server überspielt. 
Sollte auf dem Server eine andere Dateiversion verfügbar sein, als die Basis auf der der Entwickler seine Kopie herunter geladen hatte, so entsteht ein Konflikt. 
Je nach Versionierungssystem, kann dieses Konflikte zum teil automatisch lösen. 
Kann ein System die wird nun nach einer gemeinsamen Basis beider Dateien gesucht und überprüft ob es die beiden Dateien zusammenfügen kann. Wurde der Code beider Dateien an unterschiedlichen stellen geändert so können diese System den Konflikt durch zusammenführen beider Dateien selbständig beheben. Wurden die gleichen Stellen im Code geändert, so muss dies manuell durch einen der beteiligten Entwickler behoben werden. 
\end{itemize}

\section{Stand der Technik}
Die beiden aktuell am verbreitetsten Systeme sind Subversion und Git. Als Hauptgründe hierfür kann man sehen, dass beide einen sehr großen Umfang an Möglichkeiten und Flexibilität mitbringen und dadurch eines der beiden für die meisten aller zu versionierenden Projekte passend ist. Aber beiden haben ihre weite Verbreitung auch dadurch erfahren, dass sie Open-Source Projekte sind und beide bei großen Open-Source Projekten zum Einsatz kommen.

\subsection{Subversion}

\subsubsection{Geschichtliches zu Subversion}

Subversion wurde als Nachfolger für CVS, das bis dahin am weites verbreitetste Versionskontrollsystem, nicht zuletzt wegen seiner freien Verfügbarkeit und entsprechendem Einsatz in Open-Source-Projekten, entwickelt.
Im Jahr 2000 war es die amerikanische Softwarefirma CollabNet, welche bis dahin selbst CVS im Einsatz hatte. CVS hatte jedoch einige Schwachstellen, was CollabNet dazu veranlasste Kontakt zu Karl Fogel aufzunehmen und ihn zur Mitarbeit an einem neue System zu bewegen. Karl Fogel selbst hatte nicht nur ein Buch\autocite{OPSmCVS} über CVS geschrieben sondern mit seinem Freund Jim Blandy hatte er auch eine CSV Beratungs Firma gegründet. Diese war zwar bereits verkauft doch beide arbeiteten noch immer mit CVS und hatten ebenfalls schon über einen Nachfolger gesprochen und sowohl einen Namen dafür wie auch den Entwurf für das Datenablagesystem, es hieß Subversion. Im Mai 2000 war Karl Fogel angestellter bei CollabNet, zusammen mit einigen anderen Entwicklern starteten Sie mit der Entwicklung von Subversion. Es fand sich schnell Unterstützung, da viele Entwickler die gleichen Probleme mit 
CVS hatten und eine alternative suchten.

Die ersten 14 Monate lang wurde die Softwareentwicklung mit CVS verwaltet, danach waren sie mit der Entwicklung von Subversion soweit fortgeschritten, dass die Versionsverwaltung der Weiterentwicklung von Subversion auf Subversion selbst verwaltet werden konnte. 

Subversion ist kein revolutionär neues Versionsverwaltungssystem sondern vielmehr eine Evolutionäre Weiterentwicklung von CVS, was auch den Umstieg von CVS Usern zu Subversion erleichtern sollte.

Obwohl Subversion ein Open-Source-Projekt ist, finanziert CollabNet noch immer einen Großteil dessen Entwicklung und auch einige Vollzeitentwickler des Systems und hat 2009 auch alles daran gesetzt, Subversion in die Familie der Apache Software Foundation (ASF) zu integrieren. Dies gelang im Jahr 2010 auch und aus Subversion wurde Apache Subversion.

\subsubsection{Aufbau von Subversion}
Subversion gehört zu den zentralen Versionsverwaltungssystemen. Das heißt es liegt ein Repository auf einem zentralen Server und auf dieses können der oder die Entwickler zugreifen.

Auf dem Server werden die Daten des Repository entweder in einer Berkeley Datenbank oder dem Dateisystem FSFS (FileSystem-FileSystem). Dabei ist die Berkeley DB Data Store die Speicherlösung, mit der man seit beginn der Entwicklung von Subversion arbeitet. Es wurde seinerzeit ausgewählt, da es unter anderem ein stabiles und zuverlässiges System war, welches ebenfalls eine Open-Source-Lizenz hat und daher auch noch kostenlos war. Das FSFS Dateisystem wurde erst später mit der Version 1.1 eingeführt. Seit der Version 1.8 wird die Berkeley DB als veraltet betrachtet und FSFS als alleiniges Speichersystem angenommen. 

Verwendet man Subversion alleine auf einem lokalen Rechner, so ist keine Subversion Server notwendig. Sollen jedoch mehrere Entwickler über das Netzwerk auf das Repository zugriff haben, kommt man um einen Server nicht herum. Dabei sind auch hier wieder mehrere Möglichkeiten gegeben. Zum einen lässt sich auf dem Server ein Apache HTTP-Server betreiben und dann mittels HTTP Protokoll über das Netzwerk auf das Repository zugreifen. Zum anderen gibt es einen Subversion eigenen Server, den svnserve und ein dafür entwickeltest SVN Protokoll. 

Über die Kommandozeile oder verschiedene Client Anwendungen, lässt sich so mit dem Versionierungssystem arbeiten. 

\subsubsection{Die Arbeit mit Subversion}

Die einfachste Erklärung zur Arbeit mit Subversion ist, der Entwickler der mit dem Projekt arbeiten möchte, erzeugt sich eine lokale Kopie, die working copy genannt wird vom, auf dem Server liegenden Stammprojekt, welches als Trunk bezeichnet wird.Der Kopiervorgang wird als Checkout bezeichnet. Mit der lokalen Kopie arbeitet der Entwickler, hat er eine Änderungen oder Erweiterung fertig so führt er einen commit aus, bei dem alle geänderten Dateien in das Repository übertragen werden und dort für alle anderen zugänglich sind. 

So einfach ist es in der Realität kaum und daher existieren noch viele weitere wichtige Funktionen, welche für die tägliche Arbeit mit Subversion benötigt werden.

\begin{itemize}
\itemsep-8pt
\item checkout
\item repository
\item head
\item branch
\item merge
\item revision
\item tag
\item commit
\item update
\item trunk
\end{itemize}

Subversion wurde so entwickelt, dass damit mehrere Entwickler am gleichen Projekt arbeiten können. Wenn natürlich mehrere Entwickler an einem Projekt arbeiten, werden auch von mehreren die gleichen Dateien bearbeitet. Da jeder mit seiner lokalen Kopie arbeitet, stellt dies auch kein Problem dar. Werden aber nun die Änderungen von jedem Entwickler zurück gespielt, so haben die Dateien unterschiedliche Stände. Haben alle Entwickler an unterschiedlichen Stellen einer Datei Änderungen durchgeführt, so führt der Server die Änderungen zusammen. Dieser Vorgang nennt sich mergen. Wurden die Dateien jedoch an gleicher Stelle geändert, so entsteht ein Konflikt, der Commit wird mit einer Fehlermeldung abgebrochen.
Der Entwickler muss nun schauen, was die Unterschiede beider Versionen ist und diesen Konflikt beheben, es kann erforderlich sein, dass er sich mit dem anderen Entwickler in Verbindung setzt, der die Datei vor Ihm bearbeitet hat. Wurde der Konflikt behoben, muss der Commit erneut durchgeführt werden um die Dateien im Repository zu aktualisieren.

In bestimmten Fällen, kann es sein, dass Dateien nicht von mehreren Entwicklern verändert werden können. Bei Bilddateien ist dies zum Beispiel der Fall. Bei diesen Dateien gibt es keinen lesbaren Code, den das System auf Gleichheit direkt vergleichen kann bzw. entscheiden kann, ob er die Änderungen zusammenführen kann. Hier muss die ausgelagerte Datei vor der Veränderung gesperrt werden, so dass in dieser Zeit kein anderer die Datei verändern kann. Der Entwickler, der die Datei gesperrt hat, muss diese nach Bearbeitung aber wider freigeben, da er sonst die Arbeit anderer Entwickler blockieren kann.

Werden Entwicklungsarbeiten durchgeführt, welche einer längeren Entwicklungszeit bedürfen, so empfiehlt es sich, dies in einem Branch im Repository, parallel zum eigentlichen Entwicklungszweig zu tun. Dadurch hat man die Möglichkeit, das andere die Fortschritte der Entwicklung testen können, diese aber nicht direkt in den Hauptentwicklungszweig integriert sind, für den Fall, dass sie Fehler oder Inkompatibilität erzeugen.
Subversion an sich unterstützt die Entwicklung in mehreren Zweigen nicht. Für Subversion handelt es sich dabei eigentlich um eine Kopie des Trunk. Hierfür werden die Dateien nicht wirklich in ein neues Verzeichnis auf dem Server kopiert. Es werden im ersten Schritt nur Verknüpfungen zu den im Trunk befindlichen Dateien erzeugt und nur bei erfolgten Änderungen wird die Datei im neuen Verzeichnis erzeugt. Dieses Vorgehen spart Speicherplatz. Dies wird gerne als \textit{"billige Kopie"}\autocite[57]{VkmSVN} bezeichnet.

Mit der Update Funktion sollte man seine Kopie, egal ob in einem Branch oder die lokale Working Copie, immer recht aktuell halten. So hat man immer einen Überblick darüber, ob Neuerungen anderer Entwickler die eigenen beeinflussen oder umgekehrt. Es kann sonst dazu führen, dass sich die Entwicklungen zu weit von einander weg entwickeln und dann wird das Zusammenführen schwierig, wenn es zu einer Vielzahl von Konflikten kommt.

Es empfiehlt sich, immer wenn eine für den Kunden bestimmte Revision erzeugt wurde, diese zusätzlich mit einem Tag zu versehen. Dadurch wird dieser Entwicklungsstand noch einmal gesichert und man erhält die Möglichkeit diesem einen Namen zu geben, der nicht nur eine Revisionsnummer wie z. B. 3578 sondern Version 1.9 ist. Zusätzlich wird dann dieser Tag, ähnlich wie bei einem Branch, als Kopie im Repository gespeichert und ist so für spätere Zugriffe einfacher zu finden. Man kann diesen dann auch als Ausgangspunkt verwenden, wenn z.B. ein Fehler in der Software entdeckt wurde und für diese Auslieferungsversion ein passender Patch oder Release bereit gestellt werden soll.









\subsection{Git}
\subsubsection{Geschichtliches zu Git}
Die Geburtsstunde von Git war im Jahre 2005 und ergab sich aus der Not heraus. Linus Torvalds, der Erfinder von Linux benötigte für die Entwicklung von Linux ein neues Versionierungssystem. Von 2002-2005 setzte man auf BitKeeper, welches eigentlich ein kommerzielles System ist, für die Entwicklung von Linux jedoch kostenfrei war. 2005 kam es aber zu Problemen, da ein Entwickler von BitKeeper die Linux Entwickler beschuldigte, dass sie durch Reverse Engineering versuchen würden die Mechanismen von BitKeeper offen legen wollen\autocite[11]{Git}. Daraufhin wurde die Erlaubnis BitKeeper kostenfrei nutzen zu dürfen widerrufen. 
Auf dem Markt gabe es jedoch nichts, was den Ansprüchen von Linus Torvalds genügte und so beschloss er, dass eine Eigenentwicklung her musste. Für diese gab es einige Grundlegende Anforderungen\autocite[31]{PGit}:
\begin{itemize}
\itemsep-8pt
\item Geschwindigkeit
\item Einfaches Design
\item Gute Unterstützung von nicht-linearer Entwicklung (tausende paralleler Branches, d.h. verschiedener Verzweigungen der Versionen)
\item Vollständig verteilt
\item Fähig, große Projekte wie den Linux Kernel effektiv zu verwalten (Geschwindigkeit und Datenumfang)
\end{itemize}
Es dauerte nur wenige Wochen, bis Git in der Lage war sich selbst zu verwalten. Bis Git jedoch nicht nur für die Entwickler selbst bedienbar war, dauerte es bis zum Februar 2007 und der Version 1.5. Das war auch der Zeitpunkt, ab dem sich Git sehr schnell verbreitete, was nicht zuletzt auch daran lag, das Linux von einer Community von über 1000 Entwicklern stetig weiter entwickelt wird. Wie Linux wird auch Git durch eine OpenSource Community kontinuierlich weiterentwickelt.

\subsubsection{Aufbau von Git}
