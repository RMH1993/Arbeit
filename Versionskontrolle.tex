\section{Versionskontrolle}
\subsection{Anfänge der Versionsverwaltung}
Die Anfänge der Versionsverwaltung reichen bis in die 70er Jahre des vergangenen Jahrhunderts zurück. Anfang der 70er wurde bei dem amerikanischen Telekommunikationsunternehmen AT \& T eines der ersten Versionsmanagement System entwickelt, Source Code Control System. Es war für die Verwaltung einzelner Dateien ausgelegt, die nur von einem Entwickler bearbeitet wurden. Ganze Projekte und Entwicklung im Team konnte mit diesem System nicht verwaltet werden. 
Etwa 10 Jahre später wurde an der Purdue University durch Walter F. Tichy das Versionierungstool RCS entwickelt. Dabei handelte es sich wieder um ein Dateien verwaltendes Versionierungsprogramm. Dies war bereits für die Entwicklung im Team ausgelegt, vom Aufbau her jedoch so, dass wenn ein Entwickler an einer Datei gearbeitet hat diese für den Schreibzugriff durch andere Entwickler blockiert war. Beide Systeme hatten gemein, dass sie für die Versionsverwaltung lokaler Dateien waren.
Ein Jahr nach der Entwicklung von RCS wurde mit der Entwicklung von CVS (1986) begonnen. Ursprünglich als Servererweiterung für RCS gedacht, wurde das zuerst auf UNIX Shell-Skripten basierende System im Jahre 1989 in C neu programmiert. CVS verfolgte einen anderen Ansatz als seine Vorgänger, die Dateien und Projekte waren zentral auf einem Server abgelegt und die einzelnen Entwickler konnten sich ihre Arbeitskopien von der herunter laden. So konnten auch mehrere Entwickler an der gleichen Datei arbeiten. Auch war CVS in der Lage Verzeichnisstrukturen, wenn auch noch sehr eingeschränkt, zu verwalten. 
Viele der später entwickelten System bauen auf diesen Systemen auf.





\subsection{Grundlagen der Versionsverwaltung}

Der Ursprüngliche Gedanke der Versionsverwaltung war, dass man damit erfolgte Änderungen in unterschiedlichen Versionen sichtbar und differenzierbar machen konnte. Es sollte möglich sein eine Historie der erfolgten Änderungen abzubilden und für den/die Entwickler sichtbar zu machen. Noch größere Bedeutung bekam das ganze als nicht mehr nur ein Entwickler ein Thema bearbeitete sondern mehrere Entwickler im Team arbeiteten. Die Entstehung von Open-Source-Projekten mit weltweit verteilten Entwicklern hat die Entwicklung entsprechender Versionierungstools beschleunigt und deren Funktionsumfang und Herangehensweise maßgeblich beeinflusst.
Die Funktionsweise der Systeme lässt sich in Lokale, Zentrale und Verteilte Versionsverwaltung unterteilen. 
\begin{itemize}
\item{Lokale Versionsverwaltung}\\
Bei der Lokalen Versionsverwaltung waren die zu versionierenden Dateien lokal auf dem Rechner des Entwickler abgelegt. In diesen Bereich fallen auch nur die beiden zuerst entwickelten Systeme SCCS und RCS. 


\item{Zentrale Versionsverwaltung}\\
Bei der Zentralen Versionsverwaltung werden die Daten bzw. Repositoryen auf einem zentralen Server abgelegt und die Entwickler können sich von diesem Server eine lokale Arbeitskopie herunter laden und nach dem Bearbeiten wieder zurück spielen. Die Versionsgeschichte ist nur auf dem Server ersichtlich und wenn man eine Information aus einer früheren Version benötigt ist zwingend eine Verbindung zum Server notwendig. In diesen Bereich fallen zwei der am bekanntesten Open-Source-Systeme CVS und Subversion. Aber auch viele der kommerziellen Lösungen, die auf dem Markt sind, beruhen auf diesem System.



\item{Verteilte Versionsverwaltung}\\
Bei der Verteilten Versionsverwaltung besitzt jeder Entwickler lokal ein eigenes Repository des Projekts an dem er arbeitet. Die einzelnen Repository werden zwischen den unterschiedlichen Entwicklern immer wieder synchronisiert und somit alle Änderungen verteilt bzw.\@	 zusammengeführt. Ein zentraler Server ist hier nicht notwendig. Sehr häufig wird jedoch ein zentrales Repository eingesetzt um von dort aus das Produktivsystem zu bilden oder neuen Entwicklern einen Zentralen Punkt zu schaffen an dem sie sich ihre erste Arbeitskopie herunterladen können. Bei diesem System hat jeder Entwickler immer die komplette Versionsgeschichte mit in seiner Kopie des Repository und kann evtl. später entdeckte Bug's lokal in den früheren Versionen suchen und beheben.
\end{itemize}

Bei den unterschiedlichen Versionierungssystemen, gibt es auch unterschiedliche Konzepte, was die Arbeitsweise angeht. Man unterscheidet hier zwischen:
\begin{itemize}
\item Lock Modify Write\\
Bei dieser Arbeitsweise, erstellt sich der Entwickler eine Arbeitskopie der Datei, welche er ändern möchte. Das System sperrt in dieser Zeit die Datei für Schreibzugriffe anderer Entwickler. So können von einer Datei nie zwei unterschiedliche Versionen entstehen und auch keine Konflikte beim Zusammenführen der Dateien. Der Nachteil daran ist, dass nur ein Entwickler an der Datei arbeiten kann und ein anderer, der evtl. dringende Änderungen einfügen muss, dies erst tun kann, wenn der erste Entwickler die Datei wieder freigegeben hat. Ein weiterer Nachteil, wenn eine Datei gesperrt war und es bei dem Entwickler der die Datei gesperrt hatte, einen Systemverlust gab, war die Sperre noch immer vorhanden und musste umständlich entfernt werden. Bekannteste Vertreter für diese Arbeitsweise sind RCS und VSS von Microsoft. Bei den Verteilten Versionsverwaltung ist diese Arbeitsweise ausgeschlossen.

\item Copy Modify Merge\\
Bei dieser Arbeitsweise, erstellt sich jeder Entwickler seine lokale Arbeitskopie und entwickelt in dieser, hat er einen Teil fertig gestellt und will diesen den anderen zur Verfügung stellen oder ihn in das Haupt Repository übertragen, führt er einen Merge aus. Das heißt, die von ihm geänderte Datei wird auf den Server überspielt. 
Sollte auf dem Server eine andere Dateiversion verfügbar sein, als die Basis auf der der Entwickler seine Kopie herunter geladen hatte, so entsteht ein Konflikt. 
Je nach Versionierungssystem, kann dieses Konflikte zum teil automatisch lösen. 
Kann ein System die wird nun nach einer gemeinsamen Basis beider Dateien gesucht und überprüft ob es die beiden Dateien zusammenfügen kann. Wurde der Code beider Dateien an unterschiedlichen stellen geändert so können diese System den Konflikt durch zusammenführen beider Dateien selbständig beheben. Wurden die gleichen Stellen im Code geändert, so muss dies manuell durch einen der beteiligten Entwickler behoben werden. 
\end{itemize}

\section{Stand der Technik}
Die beiden aktuell am verbreitetsten Systeme sind Subversion und Git. Als Hauptgründe hierfür kann man sehen, dass beide einen sehr großen Umfang an Möglichkeiten und Flexibilität mitbringen und dadurch eines der beiden für die meisten aller zu versionierenden Projekte passend ist. Aber beiden haben ihre weite Verbreitung auch dadurch erfahren, dass sie Open-Source Projekte sind und beide bei großen Open-Source Projekten zum Einsatz kommen.
\subsection{Subversion}


\subsection{Git}

