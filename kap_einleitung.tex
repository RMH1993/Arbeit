\chapter{Einleitung}

\section{Vorwort}
Softwareentwicklung, ob für Anwendungen oder Firmeware, unterliegt einer stetigen Evolution. Neues kommt hinzu, bestehendes wird erweitert oder verbessert und Bugs werden beseitigt. 
Bei Software ist der Rhythmus, in welchem die Neuerungen und Erweiterungen zum Kunden kommen, meist durch einen vom Hersteller geplanten Produktlebenszyklus bestimmt. Nur Bugs werden in kürzeren Abständen durch Updates behoben.
Bei der Firmware für Controller ist dies häufig anders, hier bestimmen die Anforderungen von Hardwareentwicklung und Kundenwünschen die Veränderungen und Entwicklung. Wird ein Problem bei einem Kunden festgestellt, werden Fehlerbehebungen nicht mit Updates innerhalb von Wochen sondern von Stunden benötigt bzw. erwartet. Denn das Gerät, dass durch den Controller betrieben wird, muss weiter arbeiten.


\section{Untersuchungsgegenstand}
Der Untersuchungsgegenstand dieser Diplomarbeit ist die Firmware eines Präzisionsmessmaschinencontroller und die mit ihr arbeitenden Macros und SPS Programme.


\section{Ausgangspunkt}
Die Software für den Controller wurde in einem kleinen Entwicklerteam entwickelt und über Jahre hinweg immer weiter ausgebaut und verbessert. Änderungen werden stets direkt integriert und den Kundenwünschen und Anforderungen angepasst. Bugs werden so schnell wie möglich behoben und die Lösung bei den betroffenen Kunden integriert. Jeder der Entwickler hat seinen festen Aufgabenbereich und damit auch das wissen über alle Neuerungen und Fehlerbeseitigungen. Bei Überschneidungen wird zusammengearbeitet jedoch jeder in seinem Aufgabengebiet. Dadurch wurde die Dokumentation des häufigeren auch etwas stiefmütterlich behandelt und es ist teilweise auch schwierig bis überhaupt nicht nachzuvollziehen, welche Änderungen wann integriert wurden bzw. welcher Kunde dieser bereits erhalten hat. 

\section{Aufbau}

